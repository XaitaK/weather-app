\documentclass{article}
\usepackage[utf8]{inputenc}
\usepackage[russian]{babel} % Для русского языка
\usepackage{fontspec} % Подключаем fontspec для юникода
\setmainfont{Arial} % Задайте основной шрифт (например, Arial или любой другой, который поддерживает кириллицу)


\title{Документация по приложению Погоды}
\author{Николай}
\date{\today}

\begin{document}

\maketitle

\section{Введение}
В этом проекте  создал приложение для прогноза погоды с использованием HTML, CSS и JavaScript.

\section{Структура проекта}
\begin{itemize}
    \item index.html — основной файл приложения.
    \item style.css — файл стилей.
    \item app.js — файл с JavaScript-кодом.
\end{itemize}

\section{Как использовать}
1. Введите название города в текстовое поле.
2. Нажмите кнопку "Узнать погоду" для получения текущей погоды.
3. Нажмите кнопку "Прогноз на 5 дней" для получения прогноза погоды на ближайшие 5 дней.

\end{document}